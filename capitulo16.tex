\chapter{Eliminación de Producciones $\varepsilon$}
\textbf{Definición: }Denominamos producción $\varepsilon$ a una regla de la forma:
$$A\rightarrow \varepsilon$$
\textbf{Definición: }Una variable $A\in\Sigma_N$ en una GLC se dirá anulable si $A\xrightarrow{*}\varepsilon$.

\textbf{Notación: }El conjunto de todos los símbolos no terminales anulables se denotará por $N$.

\textbf{Algoritmo: }
\begin{enumerate}
\item Inicializar $N$ con todas las $A\in\Sigma_N|A\rightarrow\varepsilon$.
\item $\;$\\
\begin{tabular}{p{4cm}p{8cm}}
Repetir:	&	\\
		&Si $B\rightarrow w$; con $w\in\Sigma^*$ y todos los símbolos $w$ están en $N$, añadir $B$ a $N$.\\
Hasta que no se pueda añadir más variables a $N$		&
\end{tabular}
\end{enumerate}

\section{Algoritmo para eliminación de Reglas $\varepsilon$}
Sea $G^1=(\Sigma_N,\Sigma_T,S,P^1)$ una GLC, se obtendrá una gramática $G^2=(\Sigma_N,\Sigma_N,\Sigma_T,S,P^2)$ sin reglas $\varepsilon$, excepto $S\rightarrow \varepsilon$ tal que $L(G^1)=L(G^2)$.
\begin{enumerate}
\item Obtener $N$.
\item Inicializamos $P^2 \leftarrow \phi$.
\item Para cada regla $X\rightarrow w\in P^1$, hacer:
	\begin{itemize}
	\item Para cada representación de $w$ como $X_1Y_1X_2,...,X_nY_nX_{n+1}$ con $Y_1,...,Y_n\in N$    $P^2 \leftarrow P^2\cup \{ X\rightarrow X_1X_2...X_{n+1}\}$
	\end{itemize}
\item Devolver $G^2=(\Sigma_N,\Sigma_T,S,P^2-\{X\rightarrow\varepsilon /X\} )$
\end{enumerate}
\textbf{Ejemplo: }Sea la gramática $G^1$ con producciones $\varepsilon$:
\begin{align*}
S	&\rightarrow aA	\\
A	&\rightarrow aA|\varepsilon
\end{align*}
\textbf{Solución: }
\begin{enumerate}
\item $N=\{A\}$, todos los $A\rightarrow\varepsilon$
\item $P^2\leftarrow \phi$
\item $\,$\\
	\begin{align*}
	S	&\rightarrow aA|a	\\
	A	&\rightarrow aA|a
	\end{align*}
Luego $G^2=(\Sigma_N,\Sigma_T,S,P^2)$ donde:
	\begin{align*}
	&\Sigma_N=\{ A,S\}	\\
	&\Sigma_T=\{a\}	\\
	&P^2=\left \{ \begin{array}{l}
	S\rightarrow aA|a	\\
	A\rightarrow aA|a	
	\end{array}\right.
	\end{align*}
\end{enumerate}

\textbf{Regla: }A partir de una producción de $P^1$.
$$B\rightarrow X_1X_2...X_n\qquad X_i \mbox{ son anulables}$$
Se pueden conseguir nuevas producciones en $P^2$.

\textbf{Ejemplo: }Sea la regla $B\rightarrow X_1X_2 \mbox{ con }X_1,X_2 \mbox{ anulables}$. Se pueden obtener las producciones $B\rightarrow X_2|X_1|X_1X_2$.

\textbf{Ejemplo: }Dada la gramática:
\begin{align*}
P^1 \left \{ \begin{array}{l}
S	\rightarrow ASB|C	\\
A	\rightarrow AaaA|\varepsilon	\\
B	\rightarrow BBb|C	\\
C	\rightarrow cC|\varepsilon
\end{array}\right.
\intertext{Elimine las producciones $\varepsilon$.}
\end{align*}
\textbf{Solución: }Las variables anulables $N=\{A,C,S,B\}$, el conjunto $P^2$ está determinado por:
\begin{align*}
\left \{\begin{array}{l}
S \rightarrow SB|AB|AS|B|A|S|ASB|C	\\
A \rightarrow aaA|Aaa|aa|AaaA	\\
B \rightarrow Bb|Bb|b|C \quad\mbox{(nos quedamos con 1)}	\\
C \rightarrow c|cC
\end{array}\right.
\intertext{El conjunto $P^2$ está dado por: }
\left \{\begin{array}{l}
S \rightarrow SB|AB|AS|B|A|ASB|C|\varepsilon	\\
A \rightarrow aaA|Aaa|AaaA	\\
B \rightarrow Bb|b|C	\\
C \rightarrow c|cC
\end{array}\right.
\end{align*}
\section{Eliminación de Producciones Unitarias}
\textbf{Definición: }Una regla se llama producción unitaria o no generativa si es de la forma:
$$A\rightarrow B\qquad \mbox{con }A,B \in \Sigma_N$$
Las p.u. hacen que la GLC se vuelva compleja.

\textbf{Ejemplo: }En la GLC $G$ dada por:
\begin{align*}
A &\rightarrow B	\\
B	&\rightarrow w|C	\qquad w\in\Sigma^*	\\
\intertext{Podemos eliminar la producción }
A	&\rightarrow w|C	
\intertext{Aquí se añadió la p.u $A\rightarrow C$}
\end{align*}

\textbf{Definición: }Sea $G$ una GLC. Para cada símbolo $A\in\Sigma_N$, se define el siguiente conjunto:
$$Unit(A)=\{B\in\Sigma_N/ A\xrightarrow{*}B\}\qquad \mbox{(usando solo p.u)}$$
\textbf{OBS: } $A\in Unit(A),\mbox{ ya que } A\xrightarrow{*}A \mbox{ en $\phi$ p.u}$

\textbf{Definición: }Sean $A,B\in\Sigma_N$. Nos referiremos $(A,B)$ como el par unitario, si verifica que $A\xrightarrow{*}B$ empleando solo producciones unitarias.

\textbf{Ejemplo: }Dada la gramática $G$:
\begin{align*}
A	&\rightarrow B	\\
B	&\rightarrow C|w_1	\\
C	&\rightarrow D	\\
D	&\rightarrow w_2
\intertext{Obtener Unit(A), Unit(B), Unit(C), Unit(D)}
Unit(A)	&= \{A,B,C,D\}	\\
Unit(B)	&= \{B,C,D\}	\\
Unit(C)	&= \{C,D\}	\\
Unit(D)	&= \{D\}
\end{align*}
\subsection{Método de eliminación de p.u}
Sea $G^1=(\Sigma_N,\Sigma_T,S,P^1)$ una GLC con p.u construiremos una GLC equivalente $G^2=(\Sigma_N,\Sigma_T,S,P^2)$ sin p.u mediante:
\begin{enumerate}
\item Hállese $\Sigma_N$.
\item Para cada $A\in\Sigma_N$ determine Unit(A).
\item Para cada p.u $(A,B)$ añadir a $P^2$ todas las producciones $A\rightarrow w$ donde $B\rightarrow w$ es una producción unitaria de $P^1$.
\item Añadir a $P^2$ las producciones no unitarias restantes de $P^1$.
\end{enumerate}
\textbf{Ejemplo: }Sea $G$ la GLC con producciones unitarias:
\begin{align*}
P^1\left \{ \begin{array}{l}
S\rightarrow A|Aa	\\
A\rightarrow B	\\
B\rightarrow C|b	\\
C\rightarrow D|ab	\\
D\rightarrow b
\end{array}\right.
\intertext{Obtener la gramática equivalente $G^2$ sin p.u tal que $L(G^1)=L(G^2)$}
\end{align*}
\textbf{Solución: }
\begin{enumerate}
\item $\Sigma_N =\{ S,A,B,C,D\}$
\item $\,$\\
\begin{align*}
Unit(S)	&=\{S,A,B,C,D\}	&Unit(A)	&=\{ A,B,C,D\}	\\
Unit(B)	&=\{B,C,D\}		&Unit(C)	&=\{C,D\}	\\
Unit(D)	&=\{D\}			&
\end{align*}
\item $\,$\\
\begin{tabular}{|c|l|}\hline
Pares Unitarios	&	Producciones no unitarias	\\ \hline
(S,S)			&	$S\rightarrow Aa$	\\
(S,A)			&	-----	\\
(S,B)			&	$S\rightarrow b$	\\
(S,C)			&	$S\rightarrow ab$	\\
(S,D)			&	$S\rightarrow b$	\\
(A,A)			&	-----	\\
(A,B)			&	$A\rightarrow b$	\\
(A,C)			&	$A\rightarrow ab$	\\
(A,D)			&	$A\rightarrow b$	\\
(B,B)			&	$B\rightarrow b$	\\
(B,D)			&	$B\rightarrow ab$	\\
(B,D)			&	$B\rightarrow b$	\\
(C,C)			&	$C\rightarrow ab$	\\
(C,D)			&	$C\rightarrow b$	\\
(D,D)			&	$D\rightarrow b$	\\ \hline
\end{tabular}
\begin{align*}
\intertext{Las reglas de $P^2$ son:}
\left \{ \begin{array}{l}
S\rightarrow Aa|b|ab	\\
A\rightarrow b|ab	\\
B\rightarrow b|ab	\\
C\rightarrow ab|b	\\
D\rightarrow b
\end{array}\right.
\end{align*}
\end{enumerate}