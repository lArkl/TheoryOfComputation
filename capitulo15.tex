\chapter{Árbol de Derivación}
\textbf{Definición: }Sea $G=\{\Sigma_N,\Sigma_T,S,P\}$ una gramática libre de contexto(GLC). Un AD es un árbol ordenado, construido recursivamente como sigue:
\begin{enumerate}
\item Un árbol sin aristas cuyo único vértice tiene etiqueta $S$, es un AD de $S$.
\item Si $x\in \Sigma_N$ es etiqueta de una hoja $h$ de un AD $A$, entonces:
	\begin{itemize}
	\item Si $x\rightarrow \varepsilon \in P$, entonces el árbol obtenido incrementando a $A$ un vértice $v$ con etiqueta $\varepsilon$ y una arista $\{h,v\}$ es un AD.
	\item Si $x\rightarrow x_1x_2...x_n \in P$, donde $x_1,x_2,...x_n \in \Sigma_T \cup \Sigma_N$, entonces el árbol obtenido incrementando a $A$ $n$ vértices $v_1,v_2,...,v_n$ con etiquetas $x_1,x_2,...,x_n$ en ese orden, y con $n$ aristas $\{h,v_1\},\{h,v_2\},...\{h,v_n\}$ es un AD. 
	\end{itemize}
\end{enumerate}

\textbf{Ejemplo: }Sea $G$ un GLC donde $P$ está dado por:
\begin{align*}
& E\rightarrow E+T|T	\\
& T\rightarrow T+F|F	\\
& F\rightarrow (E)|t
\end{align*}
Obtener el AD para la cadena $w=t+(t+t)$, partiendo de E.

\textbf{Solución: }
\begin{align*}
E	&\rightarrow T			&\qquad (R_2)\\
	&\rightarrow T+F		&\qquad (R_3)\\
	&\rightarrow T+(E)		&\qquad (R_5)\\
	&\rightarrow T+(E+T)	&\qquad (R_1)\\
	&\rightarrow T+(T+T)	&\qquad (R_2)\\
\intertext{Luego se aplica las reglas reiteradas veces, uno a la vez}
	&\rightarrow F+(T+T)	&\qquad (R_4)\\
	&\rightarrow F+(F+T)	&\qquad (R_4)\\
	&\rightarrow F+(F+F)	&\qquad (R_4)\\
	&\rightarrow t+(F+F)	&\qquad (R_6)\\
	&\rightarrow t+(t+F)	&\qquad (R_6)\\
	&\rightarrow t+(t+t)	&\qquad (R_6)
\end{align*}
%GRAFICO 1
\begin{figure}[h!]
\centering
\includegraphics[width=0.3\textwidth]{img_15_1.png}
\caption{Arbol de derivación}\label{img_15_1}
\end{figure}
Se ha requerido 11 pasos para derivar $w$ (Fig \ref{img_15_1}).


\textbf{Reglas}

\begin{enumerate}
    \item La raiz se etiqueta con el simbolo inicial
    \item Los hijos de la raiz son aquellos simbolos que aparecen al lado derecho de la composicion usado para reemplazar el simbolo inicial.
    \item Todo nodo etiquetado con un no terminal tiene unos nodos hijos etiquetados en los simbolos del lado derecho de la produccion usada para sustituir ese no terminal.
    \item Los nodos que no tienene hijos deben ser etiquetados con sumbolos terminales.
\end{enumerate}


\textbf{Ejemplo: }Dada la la GLC dada por:
  \[
    P=\left\{
                \begin{array}{lll}
                    S	&\rightarrow AB		& \\
                    A	&\rightarrow aA|a	& \\
                    B	&\rightarrow bB|b	& \\
                \end{array}
              \right.
  \]
\begin{align*}
\intertext{1. Verifique si w=aabbb se deriva a partir de S}
\intertext{2. En caso afirmativo, presente el arbol de derivacion}
\intertext{\textbf{Solución: }}
S	&\rightarrow AB		&(R_1)	\\
	&\rightarrow aAB	&(R_2)	\\
	&\rightarrow aaB	&(R_3)	\\
	&\rightarrow aabB	&(R_4)	\\
	&\rightarrow aabbB	&(R_4)	\\
	&\rightarrow aabbb	&(R_5)	\\
\end{align*}
La cantidad de pesos para derviar $w$ es 6.
%graficon

Cada nodo interno del árbol será un símbolo no terminal, mientras que las hojas serán los símbolos terminales. Una regla $A:= x_1...x_n$ se representará como un símbolo cuyo nodo padre es A, siendo sus nodos hijos $x_1,x_2,...,x_n$.

\textbf{Ejemplo: }Sea $G$ la GLC dada por:
\begin{align*}
S	&\rightarrow AB		& \\
	&\rightarrow aA|a	& \\
	&\rightarrow bB|b	& \\
\intertext{Obtener el Árbol de Derivación para $w=aabbb$ e indicar la cantidad de pasos}
\intertext{\textbf{Solución: }}
S	&\rightarrow AB		&(R_1)	\\
	&\rightarrow aAB	&(R_2)	\\
	&\rightarrow aaB	&(R_3)	\\
	&\rightarrow aabB	&(R_4)	\\
	&\rightarrow aabb	&(R_5)	\\
\end{align*}
%GRAFICO 2

\begin{figure}[h!]
\centering
\includegraphics[width=0.22\textwidth]{img_15_2.png}
\caption{Arbol de derivación}\label{img_15_2}
\end{figure}
\begin{align*}
\left. \begin{array}{c}
n_i=4	\\
n_T=9
\end{array}\right \} \mbox{Pasos }=9-4=5 \quad \cmark
\end{align*}
\textbf{Definición: }Una GLC $G$ se llama ambigua, cuando es posible obtener dos o mas ADs diferentes para alguna sentencia que genere. Puede haber otras gramáticas GLC equivalente a una GLC ambigua, y que éstas gramáticas no sean ambiguas.

\textbf{Ejemplo: }Sea $G$ una GLC donde P esta dado por:
\begin{align*}
S	&\rightarrow S+S	&(R_1)	\\
	&\rightarrow S*S	&(R_2)	\\
	&\rightarrow (S)	&(R_3)	\\
	&\rightarrow t		&(R_4)	\\
\intertext{Obtener el AD para $w=t+t+t$ partiendo de $S$.}
\end{align*}
\textbf{Solución: }
\begin{align*}
S	&\rightarrow S+S	&(R_1)	\\
S	&\rightarrow S+S+S	&(R_1)	\\
S	&\rightarrow t+S+S	&(R_4)	\\
S	&\rightarrow t+t+S	&(R_4)	\\
S	&\rightarrow t+t+t	&(R_4)
\end{align*}
\begin{enumerate}
\item $n_i=5$, $n_t=10$.
%GRAFICO 3 
\begin{figure}[h!]
\centering
\includegraphics[width=0.3\textwidth]{img_15_3.png}
\caption{Arbol de derivación}\label{img_15_3}
\end{figure}
\item 
\begin{align*}
S	&\rightarrow S+S	&(R_1)	\\
S	&\rightarrow t+S	&(R_4)	\\
S	&\rightarrow t+S+S	&\mbox{ERROR }(R_1)	\\
S	&\rightarrow t+t+S	&(R_4)	\\
S	&\rightarrow t+t+t	&(R_4)
\end{align*}
%GRAFICO 4
\begin{figure}[h!]
\centering
\includegraphics[width=0.45\textwidth]{img_15_4.png}
\caption{Arbol de derivación}\label{img_15_4}
\end{figure}

\begin{align*}
S	&\rightarrow S+S	&(R_1)	\\
S	&\rightarrow S+S+S	&(R_1)	\\
S	&\rightarrow S+S+t	&(R_4)	\\
S	&\rightarrow t+S+t	&(R_4)	\\
S	&\rightarrow t+t+t	&(R_4)
\end{align*}
%GRAFICO 5
\begin{figure}[h!]
\centering
\includegraphics[width=0.45\textwidth]{img_15_5.png}
\caption{Arbol de derivación}\label{img_15_5}
\end{figure}

El (1) se puede interpretar como
$$\downlegend{t}{segundo}+(\underbrace{t+t}_{primero})$$
El (2) se puede interpretar como
$$(\underbrace{t+t}_{primero})+\downlegend{t}{segundo}$$
\end{enumerate}

\textbf{Definición: }Un lenguaje libre de contexto $L$ se dice inherentemente ambiguo si todas las GLC para $L$ son ambiguas.


\textbf{Definición: }Una derivación se denomina derivación a la izquierda si en cada paso, se expande la variable más a la izquierda.

Una derivación se dirá derivación a la derecha si en cada paso se expande la variable más a la derecha.

\textbf{Ejemplo: }Sea $G$ una GLC donde P está dado por:
\begin{align*}
S	&\rightarrow SbS	&(R_1)	\\
S	&\rightarrow ScS	&(R_2)	\\
S	&\rightarrow a		&(R_3)	\\
\end{align*}
Para la cadena $w=abaca$.
\begin{enumerate}
\item Obtener una derivación por la izquierda.
\item Obtener una derivación por la derecha.
\item Obtener su AD.
\end{enumerate}
\textbf{Solución: }
\begin{enumerate}
\item 
\begin{align*}
S	&\rightarrow \underline{S}bS	&(R_1)\\
	&\rightarrow ab\underline{S}	&(R_3)\\
	&\rightarrow ab\underline{S}cS	&(R_2)\\
	&\rightarrow abacS				&(R_3)\\
	&\rightarrow abaca				&(R_3)
\end{align*}
\item
\begin{align*}
S	&\rightarrow Sc\underline{S}	&(R_2)\\
	&\rightarrow \underline{S}ca	&(R_3)\\
	&\rightarrow Sb\underline{S}ca	&(R_1)\\
	&\rightarrow \underline{S}baca	&(R_3)\\
	&\rightarrow abaca				&(R_3)
\end{align*}
\item $\,$\\
%GRAFICO 6
\begin{figure}[h!]
\centering
\includegraphics[width=0.4\textwidth]{img_15_6.png}
\caption{Arbol de derivación}\label{img_15_6}
\end{figure}
\end{enumerate}
\section{Equivalencia en Gramáticas}
Un mismo lenguaje puede ser generado por mas de una gramática.

\textbf{Definición: }Dos gramáticas $G^1$ y $G^2$ se llaman equivalentes, si ambas generan el mismo lenguaje sobre $\Sigma_T$. Es decir:
$$L(G^1)=L(G^2)$$
Esto es, si generan el mismo lenguaje.

En muchas ocasiones es recomendable simplificar ciertas gramaticas eliminando simbolos o reglas no deseadas.

\subsubsection{Elementos Indeseables en Gramáticas}
\textbf{Definición: }Una regla innecesaria es una producción de la forma $A:=A$.

\textbf{Definición: }Sea $G=(\Sigma_N,\Sigma_T,S,P)$ una GLC. Una variable $X\in \Sigma_N$ se llama útil si y solo si existen dos cadenas $u,v\in\Sigma^*$ tales que:
\begin{align*}
S	\xrightarrow{*} uXv \mbox{ y existe } w\in{\Sigma_T}^* \mbox{ tal que } uXv\xrightarrow{*}w
\end{align*}
\textbf{Definición: }Un símbolo inaccesible o inútil es aquel símbolo no terminal que no aparece en ninguna cadena de símbolos que pueda derivarse a partir del símbolo inicial de la gramática.

\textbf{Ejemplo: }Sea $G$ la GLC dado por:
\begin{align*}
S	&\rightarrow AB|a	\\
B	&\rightarrow b	\\
C	&\rightarrow c	\\
\intertext{Identifique las variables inútiles}
\end{align*}
\textbf{Solución: }
\begin{itemize}
\item C es una variable inútil. No existen subcadenas $u,v$ tales que $S\xrightarrow{*}uCv$.
\item A es inútil?. Vemos subcadenas $u,v$ tales que $S\rightarrow AB$, $u=\varepsilon,v=B$ pero $AB\xrightarrow{*}$?. No existe $w\in{\Sigma_T}^*$, luego A es inútil.
\item B es inútil, a pesar que.
\begin{align*}
S\rightarrow AB	\\
u=A	\\
v=\varepsilon\\
\intertext{No existe w tal que }
AB\xrightarrow{*} w
\end{align*}
\end{itemize}

\textbf{Definición: }Un símbolo no generativo es aquel símbolo no terminal a partir del cual no puede derivarse ninguna cadena de símbolos terminales.

Sea $G^1 =(\Sigma_N^1, \Sigma_T, S, P^1)$ una GLC. Transformaremos $G^1$ en $G^2=(\Sigma_N^2, \Sigma_T, S, P^2)$ de modo que $L(G^1)=L(G^2)$ y para todo $A\in\Sigma_N^2$ se obtenga $A\xrightarrow{*}w$ para algún $w\in\Sigma_T^*$.

\textbf{Algoritmo}
\begin{enumerate}
\item Inicializar $\Sigma_N^2$ con las variables $A$ tales que $A\rightarrow w$ es una regla de $G^1$ donde $w\in\Sigma_T^*$
\item Inicializar $P^2$ con todas las reglas $A\rightarrow w$ para los cuales $A\in \Sigma_N^2$ y $w\in\Sigma_T^*$.
\item $\;$\\
\begin{tabular}{p{4cm}p{8cm}}
Repetir:	&	\\
		&Añadir a $\Sigma_N^2$ todas las variables $A$ para los cuales $A\rightarrow w$ para algún $w\in(\Sigma_N^2\cup\Sigma_T)^*$ que sea una producción de $P^1$ y añadirla a $P^2$.	\\
Hasta que no se puedan añadir mas variables a $\Sigma_N^2$	&
\end{tabular}		
\end{enumerate}

\textbf{Ejemplo: }Sea la gramática $G^1$:
\begin{align*}
S	&\rightarrow Aa|B|D	\\
\cmark \; B	&\rightarrow b		\\
A	&\rightarrow aA|bA|B\\
\cmark \; C	&\rightarrow abd
\end{align*}
Use el algoritmo anterior, para obtener una gramática sin símbolos no generativos.
\textbf{Solución: }
\begin{align*}
\Sigma_T	&=\{a,b,d\}	\\
\Sigma_N^1	&=\{ S,A,B,C,D\}	\\
\end{align*}
\begin{enumerate}
\item $\Sigma_N^2=\{B,C\}$
\item $P^2: \left \{ \begin{array}{l}
B\rightarrow b	\\
C\rightarrow abd
\end{array}\right.$
\item $\Sigma_N^2=\{ B,C,S,A\}$
	\begin{enumerate}
	\item $\;$\\
	\begin{align*}
	&S\rightarrow B	&\;	\\
	&A\rightarrow B	&\;	\\
	&P^2:\left \{ \begin{array}{l}
		B \rightarrow b	\\
		C \rightarrow abd	\\
		S \rightarrow B	\\
		A \rightarrow B
	\end{array}\right.	&\;
	\end{align*}
	\item $\;$\\
	\begin{align*}
	&S \rightarrow Aa	&\;	\\
	&A \rightarrow aA	&\;	\\
	&A \rightarrow bA	&\;	\\
	&P^2:\left \{ \begin{array}{l}
		B \rightarrow b	\\
		C \rightarrow abd	\\
		S \rightarrow B|Aa
	\end{array}\right.	&\;
	\end{align*}
	\item $S \rightarrow D$. Donde $D$ no está en $\Sigma_N^2$
	\end{enumerate}
\end{enumerate}