\chapter{Forma Normal de Chomsky}

\textbf{Definición: }Una $G=(\Sigma_N,\Sigma_T,s,P)$ está en la FNCh si:

\begin{itemize}
\item G no contine variables inútiles.
\item G no contiene producciones $\varepsilon$
\item G no contiene producciones unitarias.
\end{itemize}

Todas las producciones son de la forma:

\begin{align*}
A &\rightarrow \sigma \qquad\qquad A\in \Sigma_N	, \sigma \in \Sigma_T\\
A &\rightarrow BC	\qquad\qquad A,B,C \in \Sigma_N	\\
S &\rightarrow \varepsilon \qquad\qquad si\; \varepsilon\in L(G)
\end{align*}

\textbf{Teorema: }Sea $G$ una GLC. Existe una GLC en la FNCh que es equivalente a $G$.

\section{Método de Conversión}
Debemos hacer la siguientes simplificaciones:
\begin{itemize}
\item Identificas las variables anulables
\item Eliminamos las producciones $\varepsilon$ (salvo $s\rightarrow\varepsilon$)
\item Eliminamos las producciones unitarias.
\end{itemize}

Toda producción de tal gramática tendrá la forma:
$$A\rightarrow a \qquad\qquad a\in \Sigma_T$$
ó
$$A\rightarrow w \qquad\qquad |w|\geq 2$$
Luego, la tarea será:
\begin{enumerate}
\item Disponer que todas las cadenas de longitud mayor o igual a 2 consistan sólo de variables.
\item Descomponer todas las cadenas $w$ de longitud mayor o igual que 3 en una cascada de producciones tal que cada regla tenga el cuerpo formada por 2 variables.
\end{enumerate}
Para tal efecto, la construcción es como sigue:
Debemos descomponer todas las producciones de la forma:

$A\rightarrow B_1B_2...B_k$(con $k\geq 3$) en un grupo de reglas con 2 variables en el cuerpo.

Agregaremos $(k-2)$ nuevas variables: $Z_1,...,Z_{k-2}$

Con esto, la producción original se reemplaza por las $(k-1)$ reglas.

\begin{align*}
A\rightarrow B_1Z_1	\\
Z_1 \rightarrow B_2Z_2	\\
Z_2\rightarrow B_3Z_3	\\
\vdots	\\
Z_{k-2}\rightarrow B_{k-1}B_k
\end{align*}

\textbf{Ejemplo: }Reemplace las producción $A\rightarrow abBaC$ con producciones simples y binarias.

\textbf{Solución: }Incorporamos las variables $T_a,T_b$: 
\begin{align*}
T_a	&\rightarrow a	\qquad T_b\rightarrow b	\\
A	&\rightarrow T_aT_bBT_aC
\intertext{Añadimos 3 nuevas variables: $Z_1,Z_2,Z_3$}
P &= \left \{\begin{array}{l}
A\rightarrow T_aZ_1	\\
Z_1 \rightarrow T_bZ_2\\
Z_2\rightarrow BZ_3\\
Z_3\rightarrow T_aC\\
T_a\rightarrow a	\\
T_b\rightarrow b\end{array} \right.
\end{align*}

\textbf{Ejercicio: }Llevar a producciones simples y binarias la regla:
$$
A\rightarrow BAaCbb
$$
\textbf{Ejemplo: }Dada la GLC $G$
\begin{align*}
S &\rightarrow AB|aBC|SBS	\\
A &\rightarrow aA|C	\\
B &\rightarrow bbB|b	\\
C &\rightarrow cC|\varepsilon
\end{align*}
Obtener la GLC equivalente a G que esté en su FNCh.\\
\textbf{Solución: }
\begin{itemize}
\item Eliminaremos las producciones $\varepsilon$. Hallamos los anulables $\Gamma =\{C,A\}$
Agregamos las producciones:
\begin{align*}
S &\rightarrow B|aB	\\
A &\rightarrow a	\\
C &\rightarrow c
\end{align*}
Nos queda: 
\begin{align*}
G^1= \left \{ \begin{array}{l}
S\rightarrow AB|aBC|SBS	|B|aB\\
A\rightarrow aA|C|a	\\
B\rightarrow bbB|b	\\
C\rightarrow cC|c\end{array} \right.
\end{align*}
\item Eliminamos las producciones unitarias.
Para cada variables hallamos su unitario.
\begin{align*}
Unit(A)&=\{A,C\}	\\
Unit(B)&=\{B\}	\\
Unit(C)&=\{C\}	\\
Unit(S)&=\{S,B\}
\end{align*}
Incluimos los pares unitarios.

\begin{tabular}{cl}
Par no unitario - Prod. No Unitarias	\\ \hline
(A,A)	&$A\rightarrow aA|a$	\\
(A,C)	&$A\rightarrow cC|c$	\\
(B,B)	&$B\rightarrow bbB|b$	\\
(C,C)	&$C\rightarrow cC|c$	\\
(S,S)	&$S\rightarrow AB|aBC|SBS|aB$	\\
(S,B)	&$S\rightarrow bbB|b$
\end{tabular}
\begin{align*}
G^2= \left \{ \begin{array}{l}
S\rightarrow AB|aBC|SBS|aB|bbB|b	\\
A\rightarrow aA|a|cC|c	\\
B\rightarrow bbB|b	\\
C\rightarrow cC|c\end{array} \right.
\end{align*}
\item Reemplazamos los símbolos terminales por variables en cada regla que no es simple.

Incorporamos las variables: $T_a,T_b,T_c$ tales que: $T_a\rightarrow a,T_b\rightarrow b, T_c\rightarrow c$
\begin{align*}
G^3= \left \{ \begin{array}{l}
S\rightarrow AB|T_aBC|SBS|T_aB|T_bT_bB|b	\\
A\rightarrow T_aA|a|T_cC|c	\\
B\rightarrow T_bT_bB|b	\\
C\rightarrow T_cC|c\end{array} \right.
\end{align*}
Llevamos a producciones binarias incorporando $Z_1,Z_2,Z_3$
\begin{align*}
S&\rightarrow T_aBC \mbox{ se reemplaza por: } S\rightarrow T_aZ_1	\\
Z_1&\rightarrow BC\\
S&\rightarrow SBS \mbox{ se reemplaza por: } S\rightarrow SZ_2\\
Z_2&\rightarrow BS	\\
S&\rightarrow T_bT_bB \mbox{  se reemplaza por: } S_\rightarrow T_bZ_3	\\
Z_3&\rightarrow T_bB\\
B&\rightarrow T_bT_bB \mbox{ se reemplaza por: } B\rightarrow T_bZ_3	\\
Z_3&\rightarrow T_bB
\end{align*}
\begin{align*}
G^4=\left \{ \begin{array}{l}
S\rightarrow AB|T_aZ_1|SZ_2|T_aB|T_bZ_3|b	\\
A\rightarrow T_aA|a|T_cC|c	\\
B\rightarrow T_bZ_3|b	\\
C\rightarrow T_cC|c	\\
Z_1\rightarrow BC	\\
Z_2\rightarrow BS	\\
Z_3\rightarrow T_bB \\
T_a\rightarrow a	\\
T_b\rightarrow b	\\
T_c\rightarrow c \end{array} \right.
\end{align*}
$G^4$ está en FNCh
\end{itemize}